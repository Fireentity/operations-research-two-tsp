%% The first command in your LaTeX source must be the \documentclass command.
%%
%% Options you may enable if needed:
%%  twocolumn : Two-column layout.
%%  hf        : Header & footer with page numbers.
\documentclass[
% twocolumn,
% hf,
]{ceurart}

%% -------------------------------------------------
%% Extra packages you used in your draft
%% -------------------------------------------------
\usepackage{acronym}
\usepackage{subcaption}
\usepackage{changepage}
\usepackage{fancyvrb}
\DefineVerbatimEnvironment{promptbox}{Verbatim}
  {frame=single, fontsize=\small}

\renewcommand{\acffont}[1]{\textsl{#1}}
%% Minted listings support (already in template – keep it)
\usepackage{listings}
\lstset{breaklines=true}

\sloppy               % recommended by CEUR-WS
%% -------------------------------------------------
%%                FRONT  MATTER
%% -------------------------------------------------
\begin{document}

%% Rights management information.
  \copyrightyear{2025}
  \copyrightclause{Copyright for this paper by its authors.
  Use permitted under Creative Commons License Attribution~4.0
  International (CC BY 4.0).}

%% Conference information
  \conference{Graph Databases Course, University of Padua, Padua, Italy}

%% -------------------------------------------------
%% Title & subtitle
%% -------------------------------------------------
  \title{TODO}
  \title[mode=sub]{TODO}

%% (Optional) title note, can be removed if not wanted
%  \tnotemark[1]
%  \tnotetext[1]{This manuscript follows the official CEUR-WS template.}

%% -------------------------------------------------
%% Authors and affiliations
%% -------------------------------------------------
  \author[1]{Alberto Bottari}[%
    email=alberto.bottari@studenti.unipd.it
  ]

  \author[1]{Lorenzo Croce}[%
    email=lorenzo.croce.1@studenti.unipd.it
  ]\cormark[1]

%% Affiliations
  \address[1]{Department of Information Engineering, University of Padua, Via Gradenigo 6/B, 35131 Padova, Italy}

%% Footnotes
  \cortext[1]{Corresponding author.}

%% -------------------------------------------------
%% Abstract & keywords
%% -------------------------------------------------
  \begin{abstract}
    This work presents a reproducibility study of Galois, a system that treats large language models (LLMs) as a storage layer queried through SQL and natural language (NL) to obtain fully structured result tables. The study reimplements and extends the baselines and systems variants of Galois in a fully containerized Python stack.
The reproduced system is evaluated on the official internal-knowledge benchmarks using the same metrics as the original work, including F1-Cell, cardinality, and tuple-level correctness, and is tested across a broader set of modern foundation models than those originally reported.
Quantitative results show that, while SQL-based strategies do not consistently dominate NL prompting as strongly as in the original paper, the overall quality--cost trends remain comparable and highlight similar trade-offs between completeness and token usage. Beyond reproducing the baseline behavior, the implementation is designed as a modular platform that already anticipates logical and physical optimization hooks, structured schema derivation via DuckDB, and a study on the normalizing of the evaluation datasets.

  \end{abstract}

  \begin{keywords}
    Information Retrieval \sep Large Language Models  \sep Query Optimization \sep Declarative Querying \sep SQL
  \end{keywords}

  \maketitle            % build the formatted title block

%% -------------------------------------------------
%%                MAIN  CONTENT
%% -------------------------------------------------
  \input{section/introduction}
  \input{section/methodology}
  \input{section/setup}
  \input{section/results}
  \input{section/conclusion}
  \input{section/planned-improvements}
  \input{section/contributions}

%% -------------------------------------------------
%%                BACK  MATTER
%% -------------------------------------------------
%   \begin{acknowledgments}
%     % Replace or extend as needed.
%     The authors thank the organisers of CLEF 2025 LongEval for providing the data and evaluation infrastructure\ \citep{longevalCLEF2025overview}.
%   \end{acknowledgments}


%% -------------------------------------------------
%% Bibliography
%% -------------------------------------------------
  \bibliography{bibliography,proceedings}

%% -------------------------------------------------
%% Acronyms (optional)
%% -------------------------------------------------
  \newpage
  \input{acronyms}

%% -------------------------------------------------
%% End of document
%% -------------------------------------------------
\end{document}
