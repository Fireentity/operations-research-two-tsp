\section{Heuristic Methods}

Heuristic algorithms for the Traveling Salesman Problem trade optimality guarantees for speed and simplicity, and are therefore useful when instances are large or when solutions are needed under strict time constraints.
In this work, heuristics play two roles: they provide fast standalone solvers, and they serve as building blocks (e.g., warm starts or initial solutions) for more advanced approaches.
This chapter presents two construction heuristics (Nearest Neighbor and Extra-Mileage) and a local-search improvement method (2-OPT), together with a multi-start strategy used to improve robustness.

\subsection{Nearest Neighbor construction}
The Nearest Neighbor (NN) heuristic is a greedy procedure that builds a tour incrementally by repeatedly moving to the closest unvisited node until all nodes have been visited, then closing the cycle.
Its appeal lies in its extremely low overhead and deterministic behavior for a fixed start node, but solution quality can vary significantly depending on the starting city.
For this reason, NN is also used in a multi-start fashion, where the algorithm is launched from multiple starting nodes and the best tour found so far is retained.
It is important to note that our project does not provide a standalone implementation of the naive single-start Nearest Neighbor. Instead, we exclusively adopt a \textbf{Multi-Start} strategy combined with local search refinement. This approach mitigates the greedy bias by launching the algorithm from multiple distinct starting nodes and retaining only the best global tour found across all iterations.
\begin{algorithm}[H]
\caption{Nearest Neighbor Construction (Implementation)}
\SetAlgoLined
\KwIn{Distance matrix $C$, number of nodes $n$, starting node $s$}
\KwOut{Tour array $T$, total cost $cost$}

Initialize $T$ with identity $[0, 1, \dots, n-1]$\;
Swap $T[0]$ and $T[s]$\;
$current\_node \leftarrow T[0]$\;
$cost \leftarrow 0$\;

\For{$i \leftarrow 1$ \KwTo $n-1$}{
    $best\_dist \leftarrow \infty$\;
    $best\_idx \leftarrow -1$\;

    \tcp{Scan only the unvisited portion of T}
    \For{$j \leftarrow i$ \KwTo $n-1$}{
        $candidate \leftarrow T[j]$\;
        $d \leftarrow C_{current\_node, candidate}$\;
        \If{$d < best\_dist$}{
            $best\_dist \leftarrow d$\;
            $best\_idx \leftarrow j$\;
        }
    }

    $cost \leftarrow cost + best\_dist$\;
    Swap $T[i]$ and $T[best\_idx]$\;
    $current\_node \leftarrow T[i]$\;
}

$cost \leftarrow cost + C_{current\_node, T[0]}$\;
$T[n] \leftarrow T[0]$ \tcp*{Close the loop}
\Return $T, cost$\;
\end{algorithm}

\begin{algorithm}[H]
\caption{Multi-Start Nearest Neighbor + 2-OPT}
\label{alg:multistart_nn}
\SetAlgoLined
\DontPrintSemicolon

% Input e Output
\KwIn{Matrice dei costi $C$, numero di nodi $n$, tempo limite $T_{max}$}
\KwOut{Miglior tour trovato $T_{best}$, miglior costo $C_{best}$}

\BlankLine

% Inizializzazione
$S \leftarrow [0, 1, \dots, n-1]$ \tcp*{Lista dei nodi di partenza possibili}
\textbf{Shuffle} $(S)$ \tcp*{Randomizza l'ordine di esplorazione}
$C_{best} \leftarrow \infty$\;
$timer \leftarrow$ StartTimer()\;

\BlankLine

% Ciclo Multi-Start
\ForEach{start\_node $s$ in $S$}{
    \If{ElapsedTime($timer$) $> T_{max}$}{
        \textbf{break} \tcp*{Interrompe se il tempo è scaduto}
    }

    % Fase 1: Costruzione
    $T_{curr}, C_{curr} \leftarrow$ \textbf{NearestNeighborConstruct}($C, n, s$)\;

    % Fase 2: Raffinamento (Local Search)
    $improvement \leftarrow$ \textbf{TwoOpt}($T_{curr}$)\;
    $C_{curr} \leftarrow C_{curr} + improvement$\;

    % Fase 3: Aggiornamento soluzione globale
    \If{$C_{curr} < C_{best}$}{
        $C_{best} \leftarrow C_{curr}$\;
        $T_{best} \leftarrow$ Clone($T_{curr}$)\;
    }
}

\Return $T_{best}, C_{best}$\;

\end{algorithm}

\paragraph*{Parallel Execution Model}
To exploit multi-threading applied a parallelization at the \textit{Algorithm Runner} level rather than modifying the internal logic of the heuristics.
The system spawns $N$ worker threads, where each worker executes a cloned instance of the algorithm configuration with a unique random seed. This allows multiple starting points (in Multi-Start NN) or different stochastic paths (in GRASP/VNS) to be explored simultaneously.

\begin{listing}[H]
    \begin{minted}[frame=single, framesep=10pt, linenos]{c}
static void execute_parallel(const TspAlgorithm *algo,
                             const TspInstance *instance,
                             TspSolution *solution,
                             unsigned int num_threads) {
    pthread_t *threads = tsp_malloc(num_threads * sizeof(pthread_t));
    WorkerArgs *args = tsp_malloc(num_threads * sizeof(WorkerArgs));

    for (size_t i = 0; i < num_threads; i++) {
        args[i].run_fn = algo->run;
        args[i].instance = instance;
        args[i].solution = solution; // Shared solution object
        // Clone config to ensure thread-safety (e.g. separate RNG seeds)
        args[i].local_config = algo->clone_config(algo->config, i);

        pthread_create(&threads[i], NULL, worker_thread_func, &args[i]);
    }

    for (size_t i = 0; i < num_threads; i++) {
        pthread_join(threads[i], NULL);
    }
    // Cleanup...
}
    \end{minted}
    \caption{Thread creation logic in \texttt{algorithm\_runner.c} [cite: 93-99].}
    \label{lst:thread_creation}
\end{listing}

Since multiple threads attempt to update the global best solution simultaneously, access to the \texttt{TspSolution} object must be synchronized to prevent Race Conditions. We achieved this by embedding a \texttt{pthread\_mutex\_t} within the solution structure.
The update function employs a locking mechanism to ensure that a new solution is recorded only if it is strictly better than the current global best.

\begin{listing}[H]
    \begin{minted}[frame=single, framesep=10pt, linenos]{c}
bool tsp_solution_update_if_better(TspSolution *self,
                                   const int *new_tour,
                                   double new_cost) {
    bool updated = false;

    // Critical Section
    pthread_mutex_lock(&self->mutex);

    // Check if the new cost is actually better than the global best
    if (new_cost < self->cost - EPSILON) {
        self->cost = new_cost;
        memcpy(self->tour, new_tour, (n + 1) * sizeof(int));
        updated = true;
    }

    pthread_mutex_unlock(&self->mutex);

    return updated;
}
    \end{minted}
    \caption{Thread-safe solution update in \texttt{tsp\_solution.c} [cite: 997-999].}
    \label{lst:thread_safe_update}
\end{listing}

\subsection{2-OPT local search}
To improve a feasible tour produced by a construction heuristic, a standard choice is 2-OPT, which iteratively replaces two non-adjacent edges with two different edges whenever the swap reduces total length.
Given a tour \(\pi\), consider edges \((\pi_i,\pi_{i+1})\) and \((\pi_j,\pi_{j+1})\) with \(0 \le i < j-1 < n\); the 2-OPT move removes them and reconnects using \((\pi_i,\pi_j)\) and \((\pi_{i+1},\pi_{j+1})\), reversing the segment between \(i+1\) and \(j\).
The gain can be computed with the cost difference
\[
\Delta(i,j) = c(\pi_i,\pi_j) + c(\pi_{i+1},\pi_{j+1}) - c(\pi_i,\pi_{i+1}) - c(\pi_j,\pi_{j+1}),
\]
and the move is performed whenever \(\Delta(i,j) < 0\).

In our solver, 2-OPT is used as a refinement step after each constructed solution in the multi-start NN procedure, so that diversification is provided by varying the start node while intensification is obtained by local search.
The algorithm stops when no improving swap exists (local optimality with respect to the 2-OPT neighborhood) or when the global time limit is reached.

\begin{algorithm}[H]
\caption{2-OPT refinement (best-improving outline)}
\begin{algorithmic}[1]
\Require Feasible tour \(\pi\), time limit \(T\)
\Ensure Locally improved tour
\Repeat
    \State Find \((i^\star,j^\star)\) minimizing \(\Delta(i,j)\) over valid pairs.
    \If{\(\Delta(i^\star,j^\star) < 0\)}
        \State Reverse the segment \(\pi_{i^\star+1},\dots,\pi_{j^\star}\) and update the cost.
    \EndIf
\Until{\(\Delta(i^\star,j^\star) \ge 0\) or elapsed time exceeds \(T\)}
\end{algorithmic}
\end{algorithm}

\subsection{Extra-Mileage construction}
The Extra-Mileage (cheapest insertion) heuristic constructs a tour by progressively inserting new nodes into the current partial tour at the position that causes the smallest increase in total cost.
Starting from an initial subtour, at each step it evaluates candidate insertions of a not-yet-inserted node \(h\) between two consecutive nodes \(a\) and \(b\), using the extra cost
\[
\Delta(a,h,b) = c(a,h) + c(h,b) - c(a,b),
\]
and chooses the node and insertion position minimizing \(\Delta\).
This approach often produces better initial tours than purely greedy nearest-neighbor choices, at the price of a higher computational cost due to the broader evaluation at each insertion step.

\begin{algorithm}[H]
\caption{Extra-Mileage (cheapest insertion)}
\begin{algorithmic}[1]
\Require TSP instance with cost matrix \(c(\cdot,\cdot)\)
\Ensure A feasible tour \(\pi\)
\State Initialize a subtour with two nodes (then iteratively insert remaining nodes).
\While{not all nodes are inserted}
    \State Choose insertion \((a,h,b)\) minimizing \(\Delta(a,h,b)\).
    \State Insert \(h\) between \(a\) and \(b\) in the current tour.
\EndWhile
\State Close the tour and compute its cost.
\end{algorithmic}
\end{algorithm}

\subsection{Experimental comparison}
To compare heuristic variants, we evaluate them on the same benchmark instances and time limits, and we measure performance in terms of the best tour cost returned within the allowed time.
Following the experimental protocol used throughout this project, the heuristic experiments are run on 10 pseudo-random Euclidean instances with 1000 nodes each and a 60-second time limit per instance.
The comparison includes both construction heuristics alone and the same methods augmented with 2-OPT refinement; under this setup, the multi-start Nearest Neighbor combined with 2-OPT emerges as the best-performing heuristic variant on the tested instances according to the produced performance profiles.
