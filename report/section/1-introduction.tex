\section{Introduction}

The Traveling Salesman Problem (TSP) is a classical combinatorial optimization problem that has been studied extensively in operations research and computer science.
Despite its simple statement, the TSP is computationally challenging and has motivated decades of research on modeling techniques, exact optimization, and high-quality approximation methods.

This thesis presents the design, implementation, and experimental evaluation of a TSP solver developed for the Operations Research 2 course, with the goal of comparing different algorithmic paradigms under a unified and reproducible framework.
In particular, the solver includes multiple families of approaches: constructive and improvement heuristics, metaheuristics, exact methods based on integer linear programming, and matheuristic hybrids that combine mathematical programming with heuristic guidance.
The implementation focuses on making these approaches comparable by enforcing consistent data handling, identical benchmark instances, and fixed computational budgets, so that the observed differences reflect algorithmic behavior rather than engineering artifacts.


\subsection{Evaluation}

The project employs a dual-layer evaluation framework to ensure both algorithmic correctness and performance efficiency. Correctness is strictly validated through a \texttt{CTest} suite located in \texttt{tsp\_algo\_lib/tests}, which verifies the feasibility of tours and compares exact algorithm results against known optima (e.g., the \texttt{burma14} instance). Parallel to this, performance tuning is automated by a custom profiling infrastructure in the \texttt{profiler/} directory; a Python script orchestrates solver execution across a parameter grid defined in \texttt{config.json}, while a companion Bash analyzer aggregates the output CSVs to identify the most effective heuristic configurations.

This thesis is structured to guide the from theoretical foundations to the algorithmic implementations.
We begin by establishing the formal context in the next section, which defines the TSP and its standard mathematical formulation, followed by a description of the benchmarking methodology and performance profiling framework used to validate all subsequent experiments.
Building on these preliminaries, we detail the core architecture of the solver, specifically the unified data structures that ensure efficient information sharing across different modules.

The central part of the thesis explores three distinct algorithmic paradigms.
First, we discuss heuristic and metaheuristic approaches, analyzing their design choices, parameter tuning, and ability to balance exploration with exploitation.
Next, we transition to exact methods based on Integer Linear Programming, where we examine the integration with IBM ILOG CPLEX, including dynamic cut generation and warm-start strategies to prove optimality.
Finally, we bridge these two worlds with matheuristics, which embed heuristic logic directly into a MIP-based improvement loop.
Experimental results are discussed throughout the text to highlight the practical trade-offs between runtime, solution quality, and optimality guarantees on the tested instances.
