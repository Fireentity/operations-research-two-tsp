\section{Exact Methods}
\label{sec:exact_methods}

While heuristic and meta-heuristic algorithms offer a pragmatic balance between execution speed and solution quality, they cannot provide mathematical guarantees of optimality. To benchmark these approximate methods and to solve smaller instances to proven optimality, this project incorporates exact techniques based on Integer Linear Programming (ILP). This section describes the integration of an industrial-grade solver into the framework and the mathematical formulation employed.

\subsection{The Solver: IBM ILOG CPLEX}

To handle the complexity of solving NP-hard problems exactly, the project leverages **IBM ILOG CPLEX Optimization Studio**. CPLEX is widely regarded as the industry standard for solving large-scale linear and mixed-integer programming problems.

The history of this tool is deeply rooted in the evolution of computational optimization. It was originally developed in the late 1980s by Robert Bixby, a mathematician often cited as the father of modern commercial optimization. Bixby's work revolutionized the field, transforming linear programming from a theoretical mathematical exercise into a practical tool for solving real-world supply chain and logistics problems. The software was later acquired by ILOG, which was subsequently purchased by IBM in 2009, integrating the solver into its smarter analytics portfolio.

It is worth noting that the landscape of optimization solvers is heavily influenced by Bixby's legacy; after leaving IBM, he went on to co-found Gurobi Optimization, creating CPLEX's primary competitor. While numerous open-source alternatives exist (such as SCIP or the COIN-OR project solvers), proprietary solvers like CPLEX and Gurobi still maintain a significant performance edge in solving complex Mixed-Integer Programming (MIP) problems due to their advanced presolving routines and cutting-plane heuristics. Although CPLEX is a high-cost enterprise software, IBM provides the "Academic Initiative," a program that grants students and researchers free access to the full, unlimited version of the solver. This project utilizes the academic license to leverage the full power of the Branch-and-Cut engine without the constraints typically imposed on trial versions.

In the context of the Traveling Salesman Problem, CPLEX is not used as a "black box" that knows what a TSP is. Instead, it serves as a generic MIP engine: it receives a mathematical model defined by variables, objective functions, and linear constraints, and explores the search tree to find the optimal assignment of variables.

\subsection{Modeling and The CPLEX Wrapper}
\label{subsec:wrapper_modeling}

A key architectural decision in this project was to avoid using specialized TSP solvers like \textit{Concorde}. While Concorde is arguably the fastest TSP solver in existence, relying on it would have externalized the entire solution process. Instead, to maintain educational value and fine-grained control over the cutting planes, we implemented an "ad-hoc" model using the generic CPLEX C API.

To integrate this external dependency without polluting the core logic with IBM-specific data types, the system utilizes a wrapper component, \texttt{cplex\_solver\_wrapper.c}. This wrapper acts as an abstraction layer: the algorithms in \texttt{tsp\_algo\_lib} interact with a generic \texttt{CplexSolverContext}, and the wrapper translates these high-level requests (e.g., "fix this edge", "add this constraint") into low-level CPLEX calls. This design decoupling ensures that the underlying solver could theoretically be swapped (e.g., replacing CPLEX with Gurobi) without rewriting the exact algorithms.

\subsubsection{Mathematical Formulation}
The problem is modeled using the classic Dantzig-Fulkerson-Johnson (DFJ) formulation, which is generally stronger than the compact Miller-Tucker-Zemlin (MTZ) formulation for symmetric TSPs.
The decision variables are defined as binary variables $x_{ij}$, where $x_{ij} = 1$ if the edge between node $i$ and node $j$ is selected in the tour, and $0$ otherwise. Since the distance matrix is symmetric, the model considers only undirected edges ($i < j$), reducing the number of variables from $N^2$ to $N(N-1)/2$.

The base model loaded into CPLEX consists of:
\begin{enumerate}
    \item \textbf{Objective Function}: Minimize $\sum d_{ij} x_{ij}$.
    \item \textbf{Degree Constraints}: $\sum_{j} x_{ij} = 2$ for every node $i$. This ensures that every city is entered exactly once and left exactly once.
    \item \textbf{Integrality Constraints}: $x_{ij} \in \{0, 1\}$.
\end{enumerate}

Crucially, this base model (often called the \textit{2-matching relaxation}) allows for disconnected subtours (e.g., two disjoint triangles instead of one large loop). The naive approach to fix this would be to add all possible Subtour Elimination Constraints (SECs) at initialization. However, the number of SECs grows exponentially with the number of nodes ($2^N$), making it impossible to enumerate them for any non-trivial instance.

Therefore, our implementation works dynamically: the model starts with only degree constraints. Subtour constraints are identified and added "lazily" during the solution process only when a violation is detected. This dynamic row generation is the foundation of both the Benders Decomposition and the Branch-and-Cut algorithms described in the following sections.

\subsection{Benders Decomposition}
TODO

\subsection{Branch and Cut}
TODO